% Filename:      SmashTheStack.tex
% Purpose:       Which one ? 
% Authors:       CeSeNA
% License:       This file is licensed under the GPL v2.
%%%%%%%%%%%%%%%%%%%%%%%%%%%%%%%%%%%%%%%%%%%%%%%%%%%%%%%%%%%%%%%%%%%%%%%%%%%%%%%%

\author{Giacomo jake Mantani}
\title[Introduction to Smashing The Stack]{Introduction to Smashing The Stack}
\date{Dec 14 2012}
\begin{document}
\begin{frame}
        \titlepage
      \end{frame}
\begin{frame}[fragile]
\begin{verbatim}
A special thanks to CeSeNA Security group and Marco Ramilli our
"old" mentor...
\end{verbatim}
\end{frame}
\section{Term meaning}
\begin{frame}[fragile]
\begin{verbatim}

    `smash the stack` [C programming] n. On many C implementations
    it is possible to corrupt the execution stack by writing past
    the end of an array declared auto in a routine.  Code that does
    this is said to smash the stack, and can cause return from the
    routine to jump to a random address.  This can produce some of
    the most insidious data-dependent bugs known to mankind.

\end{verbatim}
\end{frame}
\section{A brief time line}
\section{31/10/1972}
\begin{frame}[fragile]
\begin{verbatim}

+ The fist document Overflow Attack (Air Force)

“By supplying addresses outside the space allocated to the users
programs, it is often possible to get the monitor to obtain unauthorized
data for that user, or at the very least, generate a set of conditions in
the monitor that causes a system crash.”

\end{verbatim}
\end{frame}
\section{2/11/1988 }
\begin{frame}[fragile]
\begin{verbatim}

+ The morris Worm

Robert Tappan Morris (Jr.) wrote and released the “Morris Worm” while still a
student at Cornell University. Aside from being the first computer worm to be
distributed via the Internet, the worm was the publics introduction to “Buffer
Overflow Attacks”, as one of the worms attack vectors was a classic stack smash
against the fingerd daemon.  In his analysis of the worm, Eugene Spafford 
writes the following: “The bug exploited to break fingerd involved overrunning
the buffer the daemon used for input. The standard C library has a few routines
that read input without checking for bounds on the buffer involved. In
particular, the gets call takes input to a buffer without doing any bounds
checking; this was the call exploited by the Worm.”
In his email message to researcher Ben Hawkes, Morris recounts his thoughts on the
overflow attack : “I had heard of the potential for exploits via overflow of
the datasegment buffers overwriting the next variable. That is, people were
worried about code like this:

        char buf[512];
        int is_authorized;
        main(){
        ...;
        gets(buf);

The idea of using buffer overflow to inject code into a program and cause it to
jump to that code occured to me while reading fingerd.c”

\end{verbatim}
\end{frame}
\section{20/10/1995 & 8/11/1996}
\begin{frame}[fragile]
\begin{verbatim}

+ "How to Write Buffer Overflow" by Peiter Zatko (mudge)

"...
    The 'Segmentation fault (core dumped)' is what we wanted to see. This
    tells us there is definately an attempt to access some memory address
    that we shouldn't. If you do much in 'C' with pointers on a unix
    machine you have probably seen this (or Bus error) when pointing or
    dereferencing incorrectly.
..."

+ "Smashing The Stack For Fun And Profit" by Elias Levy (Aleph1)

I personally think one of the best article about BoF, i used it while
i was writing this presentation.

\end{verbatim}
\end{frame}
\section{Process Memory}
\begin{frame}[fragile]
\begin{verbatim}

   To understand what stack buffers are we must first understand how a
process is organized in memory.  Processes are divided into regions like
Text, Data, and Stack.  We will concentrate on the stack region, but first
a small overview of the other regions is in order.

   The text region is fixed by the program and includes code (instructions)
and read-only data.  This region corresponds to the text section of the
executable file.  This region is normally marked read-only and any attempt to
write to it will result in a segmentation violation.

   The data region contains initialized and uninitialized data.  Static
variables are stored in this region.  The data region corresponds to the
data-bss sections of the executable file.  Its size can be changed with the
brk(2) system call.  If the expansion of the bss data or the user stack
exhausts available memory, the process is blocked and is rescheduled to
run again with a larger memory space. New memory is added between the data
and stack segments.


                             /------------------\ lower
                             |                  | memory
                             |       Text       | addresses
                             |                  |
                             |------------------|
                             |   (Initialized)  |
                             |        Data      |
                             |  (Uninitialized) |
                             |------------------|
                             |                  |
                             |       Stack      | higher
                             |                  | memory
                             \------------------/ addresses

                         Fig. 1 Process Memory
Regions

\end{verbatim}
\end{frame}
\section{Stack Region}
\begin{frame}[fragile]
\begin{verbatim}

A stack is a contiguous block of memory containing data. A register called the
stack pointer points to the top of the stack. The bottom of the stack is at a
fixed address. Its size is dynamically adjusted by the kernel at run time. The
CPU implements instructions to PUSH onto and POP off of the stack. 

   The stack consists of logical stack frames that are pushed when calling a
function and popped when returning. A stack frame contains the parameters to a
function, its local variables, and the data necessary to recover the previous
stack frame, including the value of the instruction pointer at the time of the
function call.

   Depending on the implementation the stack will either grow down (towards
lower memory addresses), or up. The stack pointer is also implementation
dependent. It may point to the last address on the stack, or to the next free
available address after the stack. 

   In addition to the stack pointer, which points to the top of the stack, it
is often convenient to have a frame pointer which points to a fixed location
within a frame. Some texts also refer to it as a local base pointer.

\end{verbatim}
\end{frame}
\end{document}
