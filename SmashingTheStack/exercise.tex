\section{Exercise}


\begin{frame}[fragile,allowframebreaks]{Tools}

\begin{block}{objdump - the linux disassembler}
		\begin{verbatim}
$ objdump -M intel -d <PROGNAME>
		\end{verbatim}
\end{block}
\framebreak		
\begin{block}{gdb - the linux debugger}
	\footnotesize
	\begin{verbatim}
$ gdb <PROGNAME>
(gdb) set disassembly-flavor intel   # we like intel sintax
(gdb) disassemble <SYMBOL-OR-ADDRESS>   # eg. disass main
(gdb) b * 0xdeadbeef			# breakpoint at address
(gdb) run <ARGS>	    # run the program
(gdb) stepi         # step into
(gdb) nexti         # step over
(gdb) finish	        # run until ret
(gdb) i r           # info registers
(gdb) i b           # info breakpoints
(gdb) x/20i	$eip	    # print 20 instr starting from EIP
(gdb) x/20w	$esp	    # 'w' WORD, 's' STRING, 'd' 
                       DECIMAL, 'b' BYTE
(gdb) display/<X-EXPR>	# like x/ but launched 
                           at every command
	\end{verbatim}
	\normalsize
\end{block}
\end{frame}

\begin{frame}[fragile,allowframebreaks]{Exercise}
	Exercises source available at \url{http://goo.gl/WupDs}\\
	Some exercises need to connect via ssh to cesena.ing2.unibo.it\\
	 as pwn at port 7357 to test your solution.\\
	(ssh pwn@cesena.ing2.unibo.it -p 7357)\\
	\begin{figure}
		\centering
		\includegraphics[height=.4\textheight]{imgs/qrcode.png}
		\caption{Exercises source}
		\label{fig:qrcode}
	\end{figure}
\framebreak
	\begin{block}{Warming up}
		\emph{auth}\\
		Just a basic overflow.\\
		Don't look too far, it's just next to you.
	\end{block}
\framebreak
	\begin{block}{Function pointer overwrite}
		\emph{nameless}\\
		Hey! A function pointer!\\
		Yes, we probably need \emph{gdb}
	\end{block}
\framebreak
	\begin{block}{Return OverWrite Easy}
		\emph{rowe}\\
		We are getting serious\\
		You'll have to OverWrite the return address!
	\end{block}
\framebreak
	\begin{block}{Return OverWrite Hard}
		\emph{rowh}\\
		Just like the previuos, but can you also prepare the data on the stack?
	\end{block}
\framebreak
	\begin{block}{Notes program}
		\emph{note}\\
		Sample notes program, ./note reads the notes, ./note "my note" adds a note\\
		You'll need a shellcode.
	\end{block}
\end{frame}
