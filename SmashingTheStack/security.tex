\section{Security}
\begin{frame}{Security Against Bofs}
	\begin{block}{How to secure the stack?}
		\begin{itemize}
			\item Various methods and techniques\ldots
			\item \ldots{}and various consideration.
			\item Which programming language?
			\item How to deal with legacy code?
			\item How to develop automatic protection?
		\end{itemize}
	\end{block}
\end{frame}

\subsection{Programming Languages}
\begin{frame}{Security: Programming Language}
	\begin{block}{Do programming languages offer automatic stack protection?}
		\begin{description}
			\item[C/C++]these languages don't provide built-int protection, but offer
				\emph{stack-safe} libraries (e.g. $strcpy() \implies strncpy()$).
			\item[Java/.NET/Perl/Python/Ruby/\ldots]all these languages provide an
				automatic array bound check: no need for the programmer to care of it.
		\end{description}
		\begin{itemize}
			\item According to \url{www.tiobe.com} C is the most used Programming Language in 2012.
			\item \alert{Legacy code still exists: it can't be rewritten!}
			\item Operating systems and compiler should offer automatic protections.
		\end{itemize}
	\end{block}
\end{frame}


\subsection{Stack Cookies}
\begin{frame}{Security: Automatic Stack Smashing detection using stack cookies}
	\begin{block}{An automatic protection introduced at compile time}
	\begin{itemize}
		\item Random words (cookies) inserted into the stack during the function prologue.
		\item Before returning, the function epilogue checks if the words are intact.
		\item If a stack smash occurs, cookie smashing is very likely to happen.
		\item The program enters in a \emph{failure} state (usually \emph{SIGSEV}).
	\end{itemize}
	\end{block}
\end{frame}

\begin{frame}{Security: StackGuard (1998)}
	\begin{block}{A patch for older gcc}
		\begin{columns}
			\begin{column}{.6\textwidth}
				\begin{itemize}
					\item ``A simple compiler technique that virtually eliminates buffer
						overflow vulnerabilities with only modest performance penalties''
						\cite{stackguard}.
					\item It offers a method for detecting return address changes in a
						portable and efficient way.
					\item StackGuard uses a random \emph{canary word} inserted before
						the return address. The callee, before returning, checks if the canary
						word is unaltered.
				\end{itemize}
			\end{column}
			\begin{column}{.4\textwidth}
				\includegraphics[width=\textwidth]{imgs/sec-canary.png}
			\end{column}
		\end{columns}
	\end{block}
\end{frame}

\begin{frame}{Security: Stack-Smashing Protector (2001)}
	\begin{block}{An improved patch for gcc}
		\begin{columns}
			\begin{column}{.5\textwidth}
		\begin{itemize}
			\item It uses a canary word (\textbf{guard}) to protect the base pointer.
			\item Relocate all arrays to the top of the stack in order to prevent variable corruption (\textbf{B} before \textbf{C}).
			\item Copies arguments into new variables below the arrays, preventing argument corruption (\textbf{A} copied into \textbf{C}).
			\item SSP is used by default since gcc 4.0 (2010), however some systems (like \emph{Arch Linux}) keep it disabled.
		\end{itemize}
			\end{column}
			\begin{column}{.5\textwidth}
				\includegraphics[width=\textwidth]{imgs/sec-ssp.png}
			\end{column}
		\end{columns}
	\end{block}
\end{frame}

\begin{frame}[shrink,fragile]{Security: SSP examples}
	\ccode
	\begin{lstlisting}
		void test(int (*f)(int), int z, char* buf) {
		  char buffer[64]; int a = f(z);
		}
	\end{lstlisting}
	\vspace{-1cm}
	\begin{columns}[T]
		\begin{column}{.5\textwidth}
			{\small \emph{\alert{gcc -m32 -fno-stack-protector test.c}}}
			\acodesmall
			\begin{lstlisting}
				push ebp
				mov  ebp,esp
				sub  esp,0x68
				mov  eax,[ebp+0xc]
				mov  [esp],eax
				mov  eax,[ebp+0x8]
				call eax
				mov  [ebp-0xc],eax
				leave
				ret
			\end{lstlisting}
		\end{column}
		\begin{column}{.5\textwidth}
			{\small \emph{\alert{gcc -m32 -fstack-protector test.c}}}
			\acodesmall
			\begin{lstlisting}
				push ebp
				mov  ebp,esp
				sub  esp,0x78
				mov  eax,[ebp+0x8]
				mov  [ebp-0x5c],eax
				mov  eax,[ebp+0x10]
				mov  [ebp-0x60],eax
				mov  eax,gs:0x14
				mov  [ebp-0xc],eax
				xor  eax,eax
				mov  eax,[ebp+0xc]
				mov  [esp],eax
				mov  eax,[ebp-0x5c]
				call eax
				mov  [ebp-0x50],eax
				mov  eax,[ebp-0xc]
				xor  eax,gs:0x14
				je   8048458 <test+0x3c>
				call 80482f0 <__stack_chk_fail@plt>
				leave
				ret
			\end{lstlisting}
		\end{column}
	\end{columns}
\end{frame}

\subsection{ASLR}
\begin{frame}[fragile]{Security: Address space layout randomization ($\sim$ 2002)}
	\begin{block}{A runtime kernel protection}
		\begin{itemize}
			\item Using PIC (position independent code) techniques and kernel aid,
				it's possible to change at every execution the position of stack, code
				and library into the addressing space.
			\item Linux implements ASLR since \emph{2.6.12}. Linux ASLR changes the
				stack position.
			\item Windows has ASLR enabled by default since Windows Vista and Windows
				Server 2008. Window ASLR changes stack, heap and Process/Thread
				Environment Block position.
		\end{itemize}
	\end{block}
\end{frame}

\begin{frame}[fragile]{Security: ASLR example}
	\begin{lstlisting}
	$ sudo sysctl -w kernel.randomize_va_space=1
	$ for i in {1..5}; do ./aslr ; done
		BP: 0x7fffe03e49d0
		BP: 0x7fff01cd44a0
		BP: 0x7fff23ac2450
		BP: 0x7fffacc72fc0
		BP: 0x7fffa20fca50
	$ sudo sysctl -w kernel.randomize_va_space=0
	$ for i in {1..5}; do ./aslr ; done
		BP: 0x7fffffffe750
		BP: 0x7fffffffe750
		BP: 0x7fffffffe750
		BP: 0x7fffffffe750
		BP: 0x7fffffffe750
	\end{lstlisting}
\end{frame}

\subsection{DEP}
\begin{frame}{Security: Data Execution Prevention ($\sim$ 2004)}
	\begin{block}{Make a virtual page not executable}
		\begin{itemize}
			\item Hardware support using the NX bit (\emph{N}ever e\emph{X}ecute)
				present in modern 64-bit CPUs or 32-bit CPUs with PAE enabled.
			\item NX software emulation techniques for older CPUs.
			\item First implemented on Linux \emph{2.6.8} and on MS Windows since
				\emph{XP SP2} and \emph{Server 2003}.
			\item Currently implemented by all OS (Linux, Mac OS X, iOS, Microsoft Windows and Android).
		\end{itemize}
	\end{block}
\end{frame}

